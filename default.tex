\documentclass[12pt, twoside]{article}
% Opciones {{{
\usepackage[pdfa, pdfusetitle, unicode=true]{hyperref}
\usepackage[spanish]{babel}
\usepackage[margin=2.5cm, a4paper]{geometry}
\usepackage{luacode}
\usepackage[shortlabels]{enumitem}
\usepackage{import}
\usepackage{xcolor}
\usepackage{fontspec}
\usepackage[mark, raisemark=0.02\paperheight, marknotags]{gitinfo2}
\usepackage{setspace}
\usepackage{cancel}

\doublespacing

% Btw I use arch
\setmonofont{InconsolataGo Nerd Font}

\newcommand{\btw}{{\color{arch}\texttt{}} }
\newcommand{\git}{{\color{git}\texttt{}} }

\renewcommand{\gitMarkPref}{{\Large\git git}}

% Esto sirve para poner ecuaciones
\usepackage{mathtools}
\usepackage{upgreek}
\allowdisplaybreaks

% Esto sirve para reducir el espacio vertical entre ecuaciones
%\AtBeginDocument{%
	%\setlength\abovedisplayskip{0pt}
	%\setlength\belowdisplayskip{0pt}}

% Esto sirve para poner imágenes{{{
\usepackage{graphicx}
\usepackage{svg}
\usepackage{subcaption}

\usepackage{float}
\usepackage{pgfplots}

\pgfplotsset{compat=1.16}
\graphicspath{ {ima/} }
%}}}
% Colores de los links {{{
\definecolor{red}{HTML}{F22C40}
\definecolor{green}{HTML}{5AB738}
\definecolor{yellow}{HTML}{D5911A}
\definecolor{blue}{HTML}{407EE7}
\definecolor{magenta}{HTML}{6666EA}
\definecolor{cyan}{HTML}{00AD9C}
\definecolor{arch}{HTML}{1793D1}
\definecolor{git}{HTML}{F54D27}

\hypersetup{
	colorlinks=true,
	linkcolor=blue,
	urlcolor=cyan,
	citecolor=magenta,
}
%}}}
% Esto controla a la cabecera {{{
\usepackage{fancyhdr}

\pagestyle{fancy}
\fancyhf{}
\renewcommand{\headrulewidth}{0pt}
\chead{ \textbf{\normalsize{Física II} }}
%\fancyhf[HL]{\includesvg[height=0.8\headheight]{Utec.svg}}
\fancyhf[FL]{\textbf{\thepage}}
\setlength{\headheight}{20pt}
\setlength{\textheight}{675pt}
%}}}
% Título {{{
\title{\textbf{Tarea}}
% Aqui hay que poner a los autores
\author{
		Alberto Oporto Ames\\
		\texttt{alberto.oporto@utec.edu.pe}\\
		%\and <++>\\
		%\texttt{<++>}\\
		}
%}}}
%}}}
% Aquí empieza el documento{{{
\begin{document}
%\maketitle
\textbf{Alberto Oporto Ames \#139}
\thispagestyle{fancy}

% Preguntas {{{
\section{Preguntas}%
\label{sec:Preguntas}

\begin{enumerate}
	\setcounter{enumi}{3}
	\item Una varilla con carga negativa se acerca a pequeños trozos de papel neutros.
		Los lados positivos de las moléculas en el papel son atraídos hacia la varilla,
		y los negativos son repelidos por ella,
		ocasionando que los trozos de papel se acerquen a la varilla.
		Si la cantidad de lados positivos y negativos es igual,
		¿por qué no se anulan entre sí las fuerzas de atracción y de repulsión?
		\subitem Porque los lados positivos están más cerca a la varilla que los lados
			neagtivos.
			Eso hace que el módulo de la fuerza de los lados positivos sea mayor
			que la de los lados negativos.

	\item Si usted camina sobre una alfombra de nailon y luego toca un objeto metálico grande,
		como una perilla, puede recibir una chispa y una descarga.
		¿Por qué esto tiende a ocurrir más en los climas secos que en los húmedos?

	\item ¿Por qué los metales son buenos conductores de la electricidad?
		\subitem Porque los átomos de los metales comparten sus electrones entre sí.
\end{enumerate}
% }}}
% Problemas {{{
\section{Problemas}%
\label{sec:Problemas}

\begin{enumerate}
	\setcounter{enumi}{2}
	\item Se tienen tres esferas conductoras de $5 cm$ de radio: $A$, $B$ y $C$.
		Se sabe que $A$ es neutra,
		mientras que $B$ tiene un exceso de $10^{11}$ electrones.
		\begin{enumerate}
			\item Determine la carga de la esfera $B$.
				\begin{align*}
					q_B &= n_Bq_e\\
					%
					q_B &= 10^{11}*-1.6*10^{-19}C\\
					%
					\Aboxed
					{
						q_B &= -1.6*10^{-8}C
					}
				\end{align*}

			\item Si se aproxima la esfera $A$ a $B$,
				¿habrá fuerza eléctrica entre ellas?
				¿Por qué?
				\subitem No.
				\subitem Porque A es neutra.

			\item Se aproxima la esfera $B$ a $C$,
				y se nota que hay una fuerza de atracción entre las dos esferas.
				¿Es posible saber el signo de la esfera $C$ con esta información?
				¿Y si la fuerza fuese de repulsión?
				\subitem Sí, es positiva $(+)$
				\subitem También, sería negativa $(-)$

			\item La esfera $B$ se pone en contacto con $A$, y luego con $C$, por separado.
				Luego de esta operación resulta que
				las cargas de $A$ y $C$ están en una relación de 4 a 1.
				Determine la carga de la esfera $C$.
				\begin{align*}
					\intertext{Al inicio}
					q_A &= 0\\
					%
					q_B &= -1.6*10^{-8}C
					\intertext{Después del primer contacto}
					q_A&=-0.8*10^{-8}C\\
					%
					q_B &=-0.8*10^{-8}C
					\intertext{Después del segundo contacto}
					q_B &= q_C\\
					%
					\frac{q_A}{q_C} &= \frac{4}{1}\\
					%
					q_C &= \frac{-0.8*10^{-8}C}{4}=\boxed{-0.2*10^{-8}C}
				\end{align*}
		\end{enumerate}

	\setcounter{enumi}{4}
	\item Dos cargas puntuales $q_1(+3\mu C)$ y $q_2(-1\mu C)$
		se encuentran ubicadas en los puntos $A(3;2;0)m$ y $B(0;-2;5)m$.
		determine:
		\begin{enumerate}
			\item La fuerza eléctrica que ejerce la carga $q_1$ sobre $q_2$.
				\begin{align*}
					\vec{F} &= k \frac{q_1q_2}{r^2} \Big{(} \frac{B-A}{r} \Big{)}\\
					%
					\vec{F} &= 9*10^9 \frac{N\cancel{m^2}}{\cancel{C^2}}*
					\frac{3*10^{-6}\cancel{C}*-1*10^{-6}\cancel{C}}{50\cancel{m^2}}*
					\Big{(} \frac{(-3\hat{\imath}-4\hat{\jmath}+5\hat{k})\cancel{m}}{\sqrt{50}\cancel{m}} \Big{)}\\
					%
					\vec{F} &= \frac{-27}{50^ \frac{3}{2} } *10^{-3} N*
					(-3\hat{\imath}-4\hat{\jmath}+5\hat{k})\\
					%
					\Aboxed
					{
						\vec{F} &= \Big{(}
						\frac{81}{50^{ \frac{3}{2} }}\hat{\imath}+
						\frac{108}{50^{ \frac{3}{2} }}\hat{\jmath}-
						\frac{135}{50^{ \frac{3}{2} }}\hat{k}
						\Big{)}*10^{-3}N
					}
				\end{align*}

			\item El campo eléctrico en el origen de coordenadas.
				\begin{align*}
					\vec{E}_t &= \vec{E}_1 + \vec{E}_2\\
					\\
					%
					\vec{E}_1 &= k \frac{q_1}{r_1^2}
					\Big{(}\frac{-A}{r_1}\Big{)}\\
					%
					\vec{E}_1 &= 9*10^{9} \frac{N\cancel{\cancel{m^2}} }{C^2}*
					\frac{3*10^{-6}C}{(13)\cancel{\cancel{m^2} }}*
					\Big{(} \frac{(-3\hat{\imath}-2\hat{\jmath})\cancel{m}}{\sqrt{13}\cancel{m}} \Big{)}\\
					%
					\vec{E}_1 &= \frac{27}{13^{ \frac{3}{2} }}*10^3\frac{N}{C}*
					(-3\hat{\imath}-2\hat{\jmath} )\\
					%
					\vec{E}_1 &= \Big{(}
					\frac{-81}{13^{ \frac{3}{2} }}\hat{\imath}-
					\frac{54}{13^{ \frac{3}{2} }} \hat{\jmath}
					\Big{)}*10^3 \frac{N}{C}\\
					\\
					%
					\vec{E}_2 &= k \frac{q_2}{r^2_2} \Big{(} \frac{-B}{r_2} \Big{)}\\
					%
					\vec{E}_2 &= 9*10^9 \frac{N\cancel{m^2}}{C^2}*
					\frac{-1*10^{-6}C}{29\cancel{m^2}}*
					\Big{(} \frac{(2\hat{\jmath}-5\hat{k})\cancel{m}}{\sqrt{29}\cancel{m}} \Big{)}\\
					%
					\vec{E}_2 &= \frac{-9*10^3}{29^{ \frac{3}{2} }}  \frac{N}{C}*
					(2\hat{\jmath}-5\hat{k})\\
					%
					\vec{E}_2 &= \Big{(}
					\frac{-18}{29^{ \frac{3}{2} }}\hat{\jmath}+
					\frac{45}{29^{ \frac{3}{2} }}\hat{k}
					\Big{)}*10^3\frac{N}{C}\\
					\\
					%
					\vec{E}_t &=  \Big{(}
					\frac{-81}{13^{ \frac{3}{2} }}\hat{\imath}-
					\frac{54}{13^{ \frac{3}{2} }} \hat{\jmath}
					\Big{)}*10^3\frac{N}{C}+
					\Big{(} \frac{-18}{29^{ \frac{3}{2} }}\hat{\jmath}+
					\frac{45}{29^{ \frac{3}{2} }}\hat{k}
					\Big{)}*10^3\frac{N}{C}\\
					%
					\Aboxed
					{
						\vec{E}_t &= \Big{(}
						\frac{-81}{13^{ \frac{3}{2} }}\hat{\imath}-
						\Big{(} \frac{54}{13^{ \frac{3}{2} }}+
						\frac{18}{29^{ \frac{3}{2} }}
						\Big{)}\hat{\jmath}+
						\frac{45}{29^{ \frac{3}{2} }}\hat{k}
						\Big{)}*10^3 \frac{N}{C}
					}
				\end{align*}

			\item La fuerza eléctrica sobre la carga $q_3(-2\mu C)$,
				que se sitúa en el origen de coordenadas.
				\begin{align*}
					\vec{F}_t &= \vec{F}_{1\_3}+\vec{F}_{2\_3}\\
					\\
					%
					\vec{F}_{1\_3} &= k \frac{q_1q_3}{r^2_{1\_3}}*
					\Big{(} \frac{-A}{r^2_{1\_3}} \Big{)}\\
					%
					\vec{F}_{1\_3} &= 9*10^9 \frac{N\cancel{m^2}}{\cancel{C^2}}*
					\frac{3*10^{-6}\cancel{C}*-2*10^{-6}\cancel{C}}{13\cancel{m^2}}*
					\Big{(} \frac{(-3\hat{\imath}-2\hat{\jmath})\cancel{m}}{\sqrt{13}\cancel{m}} \Big{)}\\
					%
					\vec{F}_{1\_3} &= \frac{-54*10^{-3}}{13^{ \frac{3}{2} }}N*
					(-3\hat{\imath}-2\hat{\jmath})\\
					%
					\vec{F}_{1\_3} &= \Big{(}
					\frac{162}{13^{ \frac{3}{2} }} \hat{\imath}+
					\frac{108}{13^{ \frac{3}{2} }} \hat{\jmath}
					\Big{)}*10^{-3}N\\
					\\
					%
					\vec{F}_{2\_3} &= k \frac{q_2q_3}{r^2_{2\_3}}*
					\Big{(} \frac{-B}{r^2_{2\_3}} \Big{)}\\
					%
					\vec{F}_{2\_3} &= 9*10^9 \frac{N\cancel{m^2}}{\cancel{C^2}}*
					\frac{-1*10^{-6}\cancel{C}*-2*10^{-6}\cancel{C}}{29\cancel{m^2}}*
					\Big{(} \frac{(2\hat{\jmath}-5\hat{k})\cancel{m}}{\sqrt{29}\cancel{m}} \Big{)}\\
					%
					\vec{F}_{2\_3} &= \frac{18*10^{-3}}{29^{ \frac{3}{2} }} N*
					(2\hat{\jmath}-5\hat{k})\\
					%
					\vec{F}_{2\_3} &= \Big{(}
					\frac{36}{29^{ \frac{3}{2} }}\hat{\jmath} -
					\frac{90}{29^{ \frac{3}{2} }}\hat{k}
					\Big{)}*10^{-3}N\\
					\\
					%
					\vec{F}_t &= \Big{(}
					\frac{162}{13^{ \frac{3}{2} }} \hat{\imath}+
					\frac{108}{13^{ \frac{3}{2} }} \hat{\jmath}
					\Big{)}*10^{-3}N+
					\Big{(}
					\frac{36}{29^{ \frac{3}{2} }}\hat{\jmath} -
					\frac{90}{29^{ \frac{3}{2} }}\hat{k}
					\Big{)}*10^{-3}N\\
					%
					\Aboxed
					{
						\vec{F}_t &= \Big{(}
						\frac{162}{13^{ \frac{3}{2} }} \hat{\imath}+
						\Big{(}
						\frac{108}{13^{ \frac{3}{2} }}+
						\frac{36}{29^{ \frac{3}{2} }}
						\Big{)}\hat{\jmath}-
						\frac{90}{29^{ \frac{3}{2} }}\hat{k}
						\Big{)}*10^{-3}N
					}
				\end{align*}
		\end{enumerate}

	\setcounter{enumi}{8}
	\item Una esfera se encuentra suspendida por medio un cable aislante
		de una barra homogénea de $80N$ de peso y longitud $L$.
		La masa de la esfera es de $5kg$ y su carga de $-4mC$.
		En toda esta región existe un campo uniforme de magnitud $5 \frac{kN}{C}$
		que apunta hacia arriba.
		Además $\sen{(\alpha)}=\frac{1}{3}$.
		Si el sistema está en equilibrio, determine:
		\begin{enumerate}
			\item La tensión en la cuerda aislante en el punto C.

			\item La tensión en la cuerda del punto B.

			\item Las reacciones en A.
		\end{enumerate}
		\begin{align*}
			\vec{E} &= \frac{5000N}{C} \hat{\jmath} && \text{Campo eléctrico}\\
			\vec{P}_1 &= -80N\hat{\jmath} && \text{Peso de la barra}\\
			\vec{P}_2 &= -50N\hat{\jmath} && \text{Peso de la esfera}\\
			\vec{F} &= -20N \hat{\jmath} && \text{Fuerza a la esfera}\\
			\vec{R} & && \text{Reacción}\\
			\vec{T}_C &= \vec{P}_2 + \vec{F} && \text{Tensión en C}\\
			\vec{T}_B & && \text{Tensión en B}\\
			\hat{AB} &= \frac{2\sqrt{2}}{3} \hat{\imath}+
			\frac{1}{3} \hat{\jmath} && \text{Vector unitario de la barra}\\
			\vec{AB} & && \text{Distancia hasta el punto B}
		\end{align*}

		\begin{enumerate}
			\item
				\begin{align*}
					\vec{T}_C &= \vec{P}_2 + \vec{F}\\
					\vec{T}_C &= -50N\hat{\jmath}-20N\hat{\jmath}\\
					\Aboxed
					{
						\vec{T}_C &= -70N\hat{\jmath}
					}
				\end{align*}
			\item
				\begin{align*}
					\sum \uptau &= \vec{P}_1\cdot\hat{AB}* \frac{L}{2}+
					\vec{T}_C\cdot\hat{AB}*L+
					\cancel{\vec{R}*0m}+
					\uptau_B=0\\
					%
					\uptau_B &= \frac{-110N}{3}L = \vec{T}_B \cdot\vec{AB}
				\end{align*}
		\end{enumerate}
\end{enumerate}
% }}}

\end{document}
%}}}
