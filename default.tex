\documentclass[12pt, twoside]{article}
% Opciones {{{
\usepackage[pdfa, pdfusetitle, unicode=true]{hyperref}
\usepackage[spanish]{babel}
\usepackage[margin=2.5cm, a4paper]{geometry}
\usepackage{luacode}
\usepackage[shortlabels]{enumitem}
\usepackage{import}
\usepackage{xcolor}
\usepackage{fontspec}
\usepackage[mark, raisemark=0.02\paperheight, marknotags]{gitinfo2}
\usepackage{setspace}

\doublespacing

% Btw I use arch
\setmonofont{InconsolataGo Nerd Font}

\newcommand{\btw}{{\color{arch}\texttt{}} }
\newcommand{\git}{{\color{git}\texttt{}} }

\renewcommand{\gitMarkPref}{{\Large\git git}}

% Esto sirve para poner ecuaciones
\usepackage{mathtools}

% Esto sirve para poner imágenes{{{
\usepackage{graphicx}
\usepackage{svg}
\usepackage{subcaption}

\usepackage{float}
\usepackage{pgfplots}

\pgfplotsset{compat=1.16}
\graphicspath{ {ima/} }
%}}}
% Colores de los links {{{
\definecolor{red}{HTML}{F22C40}
\definecolor{green}{HTML}{5AB738}
\definecolor{yellow}{HTML}{D5911A}
\definecolor{blue}{HTML}{407EE7}
\definecolor{magenta}{HTML}{6666EA}
\definecolor{cyan}{HTML}{00AD9C}
\definecolor{arch}{HTML}{1793D1}
\definecolor{git}{HTML}{F54D27}

\hypersetup{
	colorlinks=true,
	linkcolor=blue,
	urlcolor=cyan,
	citecolor=magenta,
}
%}}}
% Esto controla a la cabecera {{{
\usepackage{fancyhdr}

\pagestyle{fancy}
\fancyhf{}
\renewcommand{\headrulewidth}{0pt}
\chead{ \textbf{\normalsize{Física II} }}
%\fancyhf[HL]{\includesvg[height=0.8\headheight]{Utec.svg}}
\fancyhf[FL]{\textbf{\thepage}}
\setlength{\headheight}{40pt}
\setlength{\textheight}{675pt}
%}}}
% Título {{{
\title{\textbf{Tarea}}
% Aqui hay que poner a los autores
\author{
		Alberto Oporto Ames\\
		\texttt{alberto.oporto@utec.edu.pe}\\
		%\and <++>\\
		%\texttt{<++>}\\
		}
%}}}
%}}}
% Aquí empieza el documento{{{
\begin{document}
\maketitle
\thispagestyle{fancy}

% Preguntas {{{
\section{Preguntas}%
\label{sec:Preguntas}

\begin{enumerate}
	\setcounter{enumi}{3}
	\item Una varilla con carga negativa se acerca a pequeños trozos de papel neutros.
		Los lados positivos de las moléculas en el papel son atraídos hacia la varilla,
		y los negativos son repelidos por ella,
		ocasionando que los trozos de papel se acerquen a la varilla.
		Si la cantidad de lados positivos y negativos es igual,
		¿por qué no se anulan entre sí las fuerzas de atracción y de repulsión?

	\item Si usted camina sobre una alfombra de nailon y luego toca un objeto metálico grande,
		como una perilla, puede recibir una chispa y una descarga.
		¿Por qué esto tiende a ocurrir más en los climas secos que en los húmedos?

	\item ¿Por qué los metales son buenos conductores de la electricidad?
\end{enumerate}
% }}}
% Problemas {{{
\section{Problemas}%
\label{sec:Problemas}

\begin{enumerate}
	\setcounter{enumi}{2}
	\item

	\setcounter{enumi}{4}
	\item

	\setcounter{enumi}{8}
	\item
\end{enumerate}
% }}}

\end{document}
%}}}
